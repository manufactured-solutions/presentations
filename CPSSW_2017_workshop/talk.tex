\documentclass[mathserif]{beamer}
\usetheme[secheader]{pecostalk}
\usepackage{amsmath}
\usepackage{subfigure}
\graphicspath{{figs/}}    
\newcommand{\Lo}{\,\mathcal{L}}
\newcommand{\Diff}[2] {\dfrac{\partial( #1)}{\partial #2}}
\newcommand{\ms}{\hat{u}}
\newcommand{\bv}[1]{\ensuremath{\mbox{\boldmath$ #1 $}}}
\newcommand{\pp}[2]{\frac{\partial #1}{\partial #2}}
\newcommand{\dd}[2]{\frac{d #1}{d #2}}
\newcommand{\DD}[2]{\frac{D #1}{D #2}}
\newcommand{\mm}{\mathbf{minmod}}
\def\etal{{\it et al~}}
\newcommand{\be}{\begin{eqnarray}}
\newcommand{\ee}{\end{eqnarray}}
\newcommand{\mbb}[1]{\mathbb{#1}} % math blackboard bold
\newcommand{\mrm}[1]{\mathrm{#1}} % math Roman
\newcommand{\mcal}[1]{\mathcal{#1}} % math blackboard bold
\newcommand{\mbf}[1]{\mathbf{#1}} % math bold face (for vectors)
\newcommand{\sbf}[1]{\boldsymbol{#1}} % bold face for symbols
\newcommand{\jump}[1]{\llbracket #1 \rrbracket} % jump operator
\newcommand{\avg}[1]{\{ #1 \}} % average operator
\newcommand{\rarrow}{\rightarrow}
\newcommand{\Rarrow}{\Rightarrow}
\newcommand{\LRarrow}{\Leftrightarrow}
\newcommand{\vvvert}{|\kern-1pt|\kern-1pt|}
\newcommand{\enorm}[1]{\vvvert #1 \vvvert}
\newcommand{\nutil}{\tilde{\nu}}
\newcommand{\myred}[1]{{\color{red} #1}}
\newcommand{\sa}{\nu_{\mathrm{sa}}}
\newcommand{\brho}{\bar{\rho}}
\newcommand{\tu}{\tilde{u}}
\newcommand{\tv}{\tilde{v}}
\newcommand{\tS}{\tilde{S}}
\newcommand{\tE}{\tilde{E}}
\newcommand{\bmu}{\bar{\mu}}
\newcommand{\hh}{\tilde{h}}
\newcommand{\bp}{\bar{p}}
\newcommand{\tsa}{\mathrm{sa}}

\definecolor{Purple}{rgb}{.8,0,.8}
\definecolor{Red}{rgb}{1,0,0}
\definecolor{DarkRed}{rgb}{.5,0,0}
\definecolor{Blue}{rgb}{0,0,1}
\definecolor{DarkCyan}{rgb}{0,.6,.6}
\definecolor{DarkGreen}{rgb}{0,.5,0}

\newcommand{\data}[1]{\texttt{\color{DarkRed}#1}}
\newcommand{\comment}[1]{\texttt{\color{Red}#1}}
\newcommand{\pdv}[2]{\frac{\partial #1}{\partial #2}}
\newcommand{\Reals}{{\ensuremath{\mathbb{R}}}}
\newcommand{\Complex}{{\ensuremath{\mathbb{C}}}}
\newcommand{\Duals}{{\ensuremath{\mathbb{D}}}}
\newcommand{\Hyperduals}{{\ensuremath{\mathbb{H}}}}
\newcommand{\type}[1]{\texttt{\color{DarkGreen}#1}}
\newcommand{\omission}{{\vspace{-.7em}.........\vspace{-.3em}}}
\newcommand{\var}[1]{\texttt{\color{Blue}#1}}
\newcommand{\func}[1]{\texttt{\color{DarkCyan}#1}}
\newcommand{\key}[1]{\texttt{\color{Purple}#1}}
\newcommand{\itemdone}{\item[{\color{pecos2}\Checkmark}]}
\newcommand{\checkm}{\color{pecos2}\Checkmark}

\date{July 6th, 2017}
\author[Nicholas Malaya]{Nicholas Malaya \\
$~$ \\
{\small
AMD Research \\
Advanced Micro Devices}
}
\title[Software Verification Workshop]{Software Verification Workshop\\
and Guidelines for Scientific Software Development}

\begin{document}
\begin{frame}
  \titlepage
  \begin{flushright}
    \includegraphics[scale=0.4]{amd-research}\\
  \end{flushright}
\end{frame}

%===============================================================================
% goals of workshop
%===============================================================================
\begin{frame}[fragile]
  \frametitle{Workshop}
% This talk is online:
 % achtung! update me
  This talk is online:
\begin{verbatim}https://github.com/manufactured-solutions/presentations/\end{verbatim}

  \begin{block}{Goals}
    \begin{itemize} 
    \item Walk you through the process of code verification
    \item Build/install MASA
    \item Do a grid-refinement study for solution verification
    \item Write some code to have a little fun - do something simple
      and use MASA
    \end{itemize}    
  \end{block}

 \textcolor{red}{And throw in many ``software engineering'' comments}
\end{frame}

%===============================================================================
%New Slide - The problem statement
%===============================================================================
\begin{frame}
\frametitle{The Multiple Roles of a Computational Scientist} 

Funding agencies typically fund {\em science} and not {\em software},  therefore: \\

\begin{block}{Computational Scientists are expected to be:}
\begin{itemize}
\item Physicist -- addressed in curriculum
\item Numerical Analyst -- address in curriculum
\item Software Developer? -- seriously? 
\end{itemize}
\end{block}

{\color{pecos2}Software Engineering} (and development \ldots) {\color{pecos2}is vital to the work we do.}
\begin{itemize}
\item understandable code
\item well-documented code
\item verifiable code
\end{itemize}
\begin{center}
\small
\fcolorbox{black}{yellow}{                                                                                                                                                        
\begin{minipage}{0.85\textwidth}
Software Engineering is not frequently part of the formal curriculum; thus, most scientific software
developed in academia is unverified. Furthermore, the software developed is unable
to ever be verified without significant time investment.
\end{minipage}}
\end{center}
\end{frame}

%
%
%
% Nick to do
\begin{frame}
  \frametitle{Installing MASA locally}
  \begin{block}{Steps for Building MASA:}
    \begin{itemize}
      \item Grab latest source (https://github.com/manufactured-solutions/MASA)
	    \begin{itemize}
	     \item git clone git@github.com:manufactured-solutions/MASA.git
	    \end{itemize}
%      \item Untar: tar xvfz masa-0.41.1.tar.gz 
     \end{itemize}
   \end{block}

\begin{center}
\small
\fcolorbox{black}{yellow}{                                                                                                                                                        
\begin{minipage}{0.85\textwidth}
Key Concept: Version Control \\
 The management of changes to documents, computer programs, large web
 sites, and other collections of information.
\end{minipage}}
\end{center}
 
\end{frame}

%
%
%
%===============================================================================
%New Slide - Version Control
%===============================================================================
\begin{frame}
\frametitle{Version Control}
Minimum Guidelines -- Actually using version control is the first step

\begin{block}{Ideal Usage}
\begin{itemize}
\item Put everything under version control
\item Consider putting parts of your home directory under VC
\item Use a consistent project structure and naming convention
\item Commit often and in logical chunks
\item Write meaningful commit messages
\item Do all file operations in the VCS
\item Set up change notifications if working with multiple people
\end{itemize}
\end{block}

\begin{block}{Popular/Common Version Control Systems}
\begin{itemize}
\item Subversion - SVN (\url{http://subversion.apache.org})
%\item \href{http://subversion.apache.org}{\em Subversion (SVN)}
\item Git (\url{http://git-scm.com})
\end{itemize}
\end{block}
\end{frame}

%===============================================================================
%New Slide - Version Control
%===============================================================================
\begin{frame}
\frametitle{Version Control - Comments on Architecture}

\begin{block}{Subversion}
\begin{itemize}
\item Adopts a {\em centralized} model 
\item A local or network repisotory is defined (we have a large
  central repository that is also backed up automatically); can also
  create a repository on your own file system and back up accordingly
  (in fact, I keep several on my Dropbox account)
\item Local ``copies'' stored on your developement platform for you to
  interact with (e.g. all your fine source code)
\item Metadata stored in a .svn/ directory
\item Includes supports for binary files
\item Command line client is callen {\em svn} -- this is what you
  actually use all the time
\end{itemize}
\end{block}
\end{frame}

\begin{frame}
\frametitle{Version Control - Comments on Architecture}
\begin{block}{Git}
\begin{itemize}
\item Adopts a {\em distributed} model
\item Intended to address some short comings identified with svn for a
  large, distributed developer environment
\item Very popular in the Linux community (e.g. kernel development)
\item Excels at branching and merging
\item No requirement to have a single master source - developers can
  maintain their own personal branches and ``cherry-pick'' patches
  from another developers branch
\item Peer-to-peer pushing/pulling of code
\item Slightly higher learning curve
\end{itemize}
\end{block}
\end{frame}

\begin{frame}
\frametitle{Version Control - Controlling Your Own Destiny}
\begin{block}{Logging}
\begin{itemize}
\item A tremendous benefit of a VCS is maintaining a record over time
  of every individual change
\item This gives you, the developer, the freedom and confidence to
  add new features, implement performance refactorizations, test alternate
  algorithms, etc since you can always go back
\item Traceability is key
\item Reminder: commit early and often ({\color{pecos2} you are drastically limiting the
  potential of a VCS to help you if you only commit when you think you
  are done with a project})
\end{itemize}
\end{block}
\end{frame}

%
% git log
%
\begin{frame}[fragile]
\frametitle{Version Control - Controlling Your Own Destiny}

\begin{block}{Showing Differences (git example):}
\begin{scriptsize}
\begin{verbatim}
[nmalaya@eldritch CPSSW_2017_workshop]$ git log
commit 85ba9fe18e3b7e2e741802d5efb64411295a4e62
Author: nmalaya <nicholas.malaya@amd.com>
Date:   Wed Jul 5 12:20:57 2017 -0500

    adding figures associated with talk

commit 9e19f86650962fe9d3bc82b3a0d97af30ddc6139
Author: nmalaya <nicholas.malaya@amd.com>
Date:   Wed Jul 5 12:20:44 2017 -0500

    done with talk outline, now editing

commit c32e7c41097a9e552015d6f057255593338d60cc
Author: nmalaya <nicholas.malaya@amd.com>
Date:   Wed Jul 5 11:06:55 2017 -0500

    making spacing consistent
\end{verbatim}
\end{scriptsize}
\end{block}
\end{frame}

%
% git log
%
\begin{frame}[fragile]
\frametitle{Version Control - Controlling Your Own Destiny}

\begin{block}{Showing Differences (git example):}
\begin{scriptsize}
\begin{verbatim}
[nmalaya@eldritch CPSSW_2017_workshop]$ git diff
diff --git a/CPSSW_2017_workshop/talk.tex b/CPSSW_2017_workshop/talk.tex
index fded0f8..8d28d00 100644
--- a/CPSSW_2017_workshop/talk.tex
+++ b/CPSSW_2017_workshop/talk.tex
@@ -61,25 +61,21 @@
 \newcommand{\itemdone}{\item[{\color{pecos2}\Checkmark}]}
 \newcommand{\checkm}{\color{pecos2}\Checkmark}
 
-\date{June 30th, 2016}
+\date{July 6th, 2017}
 \author[Nicholas Malaya]{Nicholas Malaya \\
 $~$ \\
 {\small
-Center for Predictive Engineering and Computational Sciences (PECOS) \\
-Institute for Computational Engineering and Sciences (ICES) \\
-The University of Texas at Austin
+AMD Research \\
+Advanced Micro Devices}
\end{verbatim}
\end{scriptsize}
\end{block}
\end{frame}

% Nick to do
\begin{frame}
  \frametitle{Installing MASA locally}
  \begin{block}{Steps for Building MASA:}
    \begin{itemize}
      \item Grab latest source (https://github.com/manufactured-solutions/MASA)
	    \begin{itemize}
	     \item git clone git@github.com:manufactured-solutions/MASA.git
	    \end{itemize}
%      \item Untar: tar xvfz masa-0.41.1.tar.gz 
      \item Configure: ./configure --prefix=\$HOME/masa % what about compilers here?
      \item Compile: make -j 2
     \end{itemize}
   \end{block}

\begin{center}
\small
\fcolorbox{black}{yellow}{                                                                                                                                                        
\begin{minipage}{0.85\textwidth}
Key Concept: Build System
\end{minipage}}
\end{center}

\end{frame}

%
%
%
%===============================================================================
%New Slide - Build System
%===============================================================================
\begin{frame}
\frametitle{Build Systems}

\begin{block}{Minimum Guidelines} 
\begin{itemize}
\item Any build system is a good start but it's more than an alias
\item Bash script
\item Makefile
\item Host specific Makefiles
\end{itemize}
\end{block}

\begin{block}{A proper build system}
\begin{itemize}
\item Cross platform support
\item Runtime checks for local functionality
\item Supports dependencies
\item Supports finding include files and libraries
\item Supports parallel builds
\end{itemize}
\end{block}

{\color{pecos2} Common build tools are Make, autoconf/automake,  CMake, SCons}
\end{frame}

%
%
%
% Nick to do
\begin{frame}
  \frametitle{Installing MASA locally}
  \begin{block}{Steps for Building MASA:}
    \begin{itemize}
      \item Grab latest source (https://github.com/manufactured-solutions/MASA)
	    \begin{itemize}
	     \item git clone git@github.com:manufactured-solutions/MASA.git
	    \end{itemize}
%      \item Untar: tar xvfz masa-0.41.1.tar.gz 
      \item Configure: ./configure --prefix=\$HOME/masa % what about compilers here?
      \item Compile: make -j 2
      \item Test: make check
     \end{itemize}
   \end{block}
\begin{center}
\small
\fcolorbox{black}{yellow}{                                                                                                                                                        
\begin{minipage}{0.85\textwidth}
Key Concept: Unit Testing
\end{minipage}}
\end{center}

\end{frame}

%
%
%
%===============================================================================
%New Slide - Testing
%===============================================================================
\begin{frame}
\frametitle{Trust No One \ldots not even yourself}
\begin{block}{Unit Testing}
\begin{footnotesize}
\begin{itemize}
\item Tests individual units of source code 
\item A Unit is the smallest testable part of a code
\item Each Unit test should be independent from others
\item Best to write the Unit and the Unit test at the same time
\end{itemize}
\end{footnotesize}
\end{block}

\begin{block}{Regression Testing}
\begin{footnotesize}
\begin{itemize}
\item Examples, applications, unit tests, benchmark problems
\item Catches unintended consequences of revisions
\end{itemize}
\end{footnotesize}
\begin{columns}[c]
\begin{column}{7cm}
\begin{footnotesize}
\begin{itemize}
\item Test early and test often
\item Consider automating tests 
\end{itemize}
\end{footnotesize}
\end{column}
\begin{column}{4.7cm}
\includegraphics[width=.9\linewidth]{trust}
\end{column}
\end{columns}
\end{block}
\end{frame}


%===============================================================================
% refresher
%===============================================================================
\begin{frame}
  \frametitle{Problem: Solve 2D Laplacian using Finite-Differencing}
  \begin{columns}[c]
    \begin{column}{5cm}

      \includegraphics[width=1\linewidth]{domain}

    \end{column}
    
    \begin{column}{6.5cm}
      
      \begin{block}{Recall:}
        \begin{itemize} 

          \item Laplace's Equation in 2D:
          \begin{equation}
            \nonumber     
            \frac{\partial^2 \phi}{\partial x^2} + \frac{\partial^2 \phi}{\partial y^2} = 0
          \end{equation}

	  \item For the verification exercise, we will replace the RHS
	    above with a forcing function $f(x,y)$ that we get from
	    MASA
          \begin{equation}
            \nonumber     
            \frac{\partial^2 \phi}{\partial x^2} + \frac{\partial^2
	      \phi}{\partial y^2} = f(x,y)
	    \end{equation}
          
          % nick: not sure about this here, do we want to show mms?
          % seems less useful to remind them about the source term
%%           \item Our manufactured solution is:
%%           \begin{equation}
%%             \begin{split}
%%               \nonumber
%%               \phi(x,y) &= (\textcolor{blue}{Ly}-y)^2 (\textcolor{blue}{Ly}+y)^2 \\ 
%%               &+ (\textcolor{red}{Lx}-x)^2 (\textcolor{red}{Lx}+x)^2
%%             \end{split}
%%           \end{equation}          
%%           \item By default, \textcolor{blue}{Ly}$=0.82$ and \textcolor{red}{Lx}$=1.4$

        \end{itemize}    
      \end{block}
      
    \end{column}
  \end{columns}

%%   % karl: can you check that this is the differencing scheme you intended?
%%   Solve using forward difference:
%%   \begin{equation}
%%     \nonumber     
%%     \phi_i'' \approx \frac{\phi_{i+2} - 2\phi_{i+1} + \phi_{i}}{h^2} + O(h^2)
%%   \end{equation}
  
\end{frame}

 \begin{frame}
   \frametitle{Problem: Solve 2D Laplacian using Finite-Differencing}

   \begin{block}{Outline}
     \begin{itemize} 
     \item {\em Goal:} Write a program in C/C++, F90, python, or Matlab/Octave
       which solves the two-dimensional Laplacian on a square domain
     \item {\em Inputs}: 
       \begin{itemize}
	 \item \# of points in one direction ({\em npts})
	 \item the physical dimension of one side ({\em length})
       \end{itemize}
     \item {\em Output}: $l_2$ error between your numerical solution
       and an exact solution derived from a manufactured solution in
       MASA
       \begin{equation}
         \nonumber
         l_2 = \sqrt{ \frac{\sum_{i=1}^{\text{\tiny N}} (\phi_i-\phi_i^{\text{\tiny exact}})^2}N}
       \end{equation}
     \item {\em Runs}: Run your snazzy code for $\text{npts} =
       5,9,17,\text{and } 33$ and plot $l_2$ norm as a function of $1/h$ where
       $h=\text{length}/(\text{npts}-1)$
       
     \end{itemize}    
   \end{block}

 \end{frame}

\begin{frame}
  \frametitle{Finite-difference Scheme}
    \begin{block}{Method}
      \begin{itemize} 
	\item Let us use a simple FD approximation for the Laplacian
        \item Assume a constant spacing mesh for convenience
        \item Central-differencing 
	  \begin{equation}
	    \nonumber     
	    \nabla^{2}{\phi}_{i,j} \approx \frac{\phi_{i+1,j} -
	      2\phi_{i,j} + \phi_{i-1,j}}{h^2} + 
	    \frac{\phi_{i,j+1} -
	      2\phi_{i,j} + \phi_{i,j-1}}{h^2} + O(h^2)
	  \end{equation}
	  \item Use this formula to build the coefficient entries into
	    a linear system $Ax=b$.  
	  \item The size of the linear system is the number of solution
	    points. Since we are on a square domain,  $N = \text{npts}*\text{npts}$
      \end{itemize}
    \end{block}
\end{frame}

\begin{frame}
  \frametitle{Finite-difference Scheme}
    \begin{block}{Boundary Conditions}
      \begin{itemize} 
          \item The 5-point FD stencil is incomplete on the boundaries of
	    our square domain 
	  \item We need to apply constraints to matrix
	    $A$ to enforce the Dirchlet conditions on the boundaries
	  \item Simplest method to enforce BCs:
	    \begin{itemize} 
	      \item zero out all matrix entries on the
		row associated with boundary point, $\phi_i$
	    \item set the
	      diagonal $A(i,i)$ = 1.0
	      \item Set the RHS function to the
		desired solution $f_i = \phi_{\text{exact}}$
		\end{itemize}
      \end{itemize}
    \end{block}
\end{frame}

\begin{frame}[fragile]
  \frametitle{Finite-difference Scheme}
    \begin{block}{Boundary Conditions}
      \begin{itemize} 
          \item Let's look at form of system matrix $A^*$ after BCs have
	    been applied for $\text{npts}=3$ (note $A=\frac{1}{h^2}A^*$):
{\tiny
\begin{verbatim}


     j =     0      1      2      3      4      5      6      7      8 

i =  0:   1.00   0.00   0.00   0.00   0.00   0.00   0.00   0.00   0.00 
i =  1:   0.00   1.00   0.00   0.00   0.00   0.00   0.00   0.00   0.00 
i =  2:   0.00   0.00   1.00   0.00   0.00   0.00   0.00   0.00   0.00 
i =  3:   0.00   0.00   0.00   1.00   0.00   0.00   0.00   0.00   0.00 
i =  4:   0.00   1.00   0.00   1.00  -4.00   1.00   0.00   1.00   0.00 
i =  5:   0.00   0.00   0.00   0.00   0.00   1.00   0.00   0.00   0.00 
i =  6:   0.00   0.00   0.00   0.00   0.00   0.00   1.00   0.00   0.00 
i =  7:   0.00   0.00   0.00   0.00   0.00   0.00   0.00   1.00   0.00 
i =  8:   0.00   0.00   0.00   0.00   0.00   0.00   0.00   0.00   1.00 


\end{verbatim}
}
\item In this case, we only have {\em one} active interior solution point
      \end{itemize}
    \end{block}
\end{frame}

%===============================================================================
% application linkage
%===============================================================================
\begin{frame}[fragile]
\frametitle{Fortran 90 Reminder: What you need from MASA}
{\tiny
\begin{verbatim}
program main
  use masa
  implicit none

  dx = real(lx)/real(nx)
  dy = real(ly)/real(ny);

  ! initialize the problem
  call masa_init("laplace example","laplace_2d")

  ! evaluate source terms (2D)
  do i=0, nx
     do j=0, ny
         
        y = j*dy        
        x = i*dx
        
        ! evalulate source term
        field = masa_eval_2d_source_f   (x,y)

        ! evaluate analytical term
        exact_phi = masa_eval_2d_exact_phi (x,y)

     enddo
  enddo

end program main

\end{verbatim}
}
\end{frame}

%===============================================================================
% MASA API demo
%===============================================================================
\begin{frame}[fragile]
\frametitle{C Reminder: What you need from MASA}
{\tiny
\begin{verbatim}
#include <masa.h>

int main()
{
  err += masa_init("laplace example","laplace_2d");

  // grab / set parameter values
  Lx = masa_get_param("Lx");
  masa_set_param("Ly",42.0);

  for(int i=0;i<nx;i++)
    for(int j=0;j<nx;j++)
      {  
        x=i*dx;
        y=j*dy;

        // source term
        ffield    = masa_eval_2d_source_f (x,y);

        // manufactured solution
        phi_field = masa_eval_2d_exact_phi(x,y);

       } // finished iterating over space
} //end program
\end{verbatim}
}
\end{frame}
	   
% Nick to do
\begin{frame}
  \frametitle{Installing MASA locally}
  \begin{block}{Steps for Building MASA:}
    \begin{itemize}
      \item Grab latest source (https://github.com/manufactured-solutions/MASA)
	    \begin{itemize}
	     \item git clone git@github.com:manufactured-solutions/MASA.git
	    \end{itemize}
%      \item Untar: tar xvfz masa-0.41.1.tar.gz 
      \item Configure: ./configure --prefix=\$HOME/masa % what about compilers here?
      \item Compile: make -j 2
      \item Test: make check
      \item Install locally: make install
      \item To generate documentation: make docs 
	\begin{itemize}
	  \item Can then point a broswer to docs/html/index.html
	\end{itemize}
     \end{itemize}
   \end{block}

\end{frame}

\begin{frame}[fragile]
  \frametitle{Linking to your installed MASA}
  \begin{block}{Linking against your local build}
    \begin{itemize} 

      \item {\bf C}: Assuming your code is named laplacian.c and you
      installed masa into \$HOME/masa: 
{\tiny
\begin{verbatim}

gcc -I$HOME/masa/include laplacian.c -L$HOME/masa/lib -lmasa

\end{verbatim}
}
      
      \item {\bf F90}: Assuming your code is named laplacian.f90 
{\tiny
\begin{verbatim}

gfortran -I$HOME/masa/lib laplacian.f90 -L$HOME/masa/lib -lmasa -lfmasa

\end{verbatim}
}
      \end{itemize}
    \end{block}
\end{frame}

%===============================================================================
% application linkage
%===============================================================================
\begin{frame}
  \frametitle{General Verification Approach Using MMS and MASA}
  \begin{center}
    \includegraphics[width=.8\linewidth]{masa_overview} \\
  \end{center}
\end{frame}


\begin{frame}[fragile]
  \frametitle{General Program Flow (a C example)}
{\tiny
%  \begin{center}
  \begin{verbatim}
int main(int argc, char *argv[])
{
  int n;
  double length;
  pstruct model;                /* primary model data structure */

  /* Parse command-line */

  if(argc < 2)
    {
      printf("\nUsage: laplacian [num_pts] [length]\n\n");
      printf("where \"num_pts\" is the desired number of mesh points and \n");
      printf("\"length\" is the physical length-scale dimension in one direction\n\n");
      
      exit(1);
    }
  else
    {
      n      = atoi(argv[1]);
      length = (double) atof(argv[2]);
    }

  /* Problem Initialization */

  problem_initialize (n,length,&model);
  assemble_matrix    (1,&model);
  init_masa          (&model);
  apply_bcs          (&model);

  .....
\end{verbatim}
%  \end{center}
}  
\end{frame}

\begin{frame}[fragile]
  \frametitle{General Program Flow (a C example, continued)}
{\tiny
  \begin{verbatim}

  /* Solve linear system */

  solve_gauss       (&model);

  /* Compute Error */

  printf("\n** Error Analysis\n");
  printf("   --> npts     = %i\n",model.npts);
  printf("   --> h        = %12.5e\n",model.h);
  printf("   --> l2 error = %12.5e\n",compute_l2_error(&model));

  return 0;

}
\end{verbatim}
}  
\end{frame}


%% more code foo
\begin{frame}[fragile]
  \frametitle{Example Model Data Structure}
{\tiny
  \begin{verbatim}

typedef struct pstruct {
  double  *phi;                 /*!< solution variable                    */
  double  *rhs;                 /*!< right-hand side forcing function     */
  double **A;                   /*!< linear system matrix                 */
  double   h;                   /*!< mesh sizing                          */
  int      n;                   /*!< problem size                         */
  int      npts;                /*!< number of points in single direction */
  int      pad;                 /*!< pad dimension for ghost points       */
} pstruct;

\end{verbatim}
}  
\end{frame}

%===============================================================================
% slide where we show an example of just that, using Karls plot
%===============================================================================
\begin{frame}
  \frametitle{Example Results: What we're hoping for}
  2nd Order Central Finite-difference Scheme

\begin{center}
\begin{columns}[c]
\begin{column}{6cm}
    \includegraphics[width=0.95\linewidth]{laplace_central_diff1}
\end{column}
\begin{column}{6cm}
    \includegraphics[width=0.95\linewidth]{laplace_central_diff2}
\end{column}
\end{columns}
\begin{itemize}
\item Example results for $\text{npts} = 5,9,17,33,65$, $\text{length} = 1.0$
\end{itemize}
  \end{center}
  
\end{frame}

%===============================================================================
% NEW SLIDE: finis
%===============================================================================
\begin{frame}[fragile]
\frametitle{MASA PDE Examples}

\begin{block}{Source Terms: Euler}
\begin{semiverbatim}
\footnotesize

\comment{// Gas state}
\type{ADScalar} \var{T}  = \var{P} / \var{RHO} / \var{R};
\type{ADScalar} \var{E}  = \data{1.} / (\var{Gamma}-\data{1.}) * \var{P} / \var{RHO};
\type{ADScalar} \var{ET} = \var{E} + \data{.5} * \var{U}.\func{dot}(\var{U});

\comment{// Mass, momentum and energy}
\type{Scalar}   \var{Q_rho}   = \func{raw_value}(\func{divergence}(\var{RHO}*\var{U}));
\type{RawArray} \var{Q_rho_u} = \func{raw_value}(\func{divergence}(\var{RHO}*\var{U}.\func{outerproduct}(\var{U})) +
                   \var{P}.\func{derivatives}());
 \type{Scalar}  \var{Q_rho_e} = \func{raw_value}(\func{divergence}((\var{RHO}*\var{ET}+\var{P})*\var{U}));

\end{semiverbatim}
\end{block}
 {\bf check out tests/ad\_euler.cpp}
\end{frame}

%===============================================================================
% NEW SLIDE: finis
%===============================================================================
\begin{frame}
  \frametitle{}
  \begin{block}{}
    \center{Thank you for your attention.} \\
    \center{Let's start coding!!!}
  \end{block}
\end{frame}

% %===============================================================================
% % goals of workshop
% %===============================================================================
% \begin{frame}
%   \frametitle{Workshop}

%   \begin{block}{Goals}
%     \begin{itemize} 
%     \item Walk you through the process of code verification
%     \item Build/install MASA
%     \item Do a grid-refinement study for solution verification
%     \item Write some code to have a little fun - do something simple
%       and use MASA
%     \end{itemize}    
%   \end{block}

% \end{frame}

% %===============================================================================
% % refresher
% %===============================================================================
% \begin{frame}
%   \frametitle{Problem: Solve 2D Laplacian using Finite-Differencing}
%   \begin{columns}[c]
%     \begin{column}{5cm}

%       \includegraphics[width=1\linewidth]{domain}

%     \end{column}
    
%     \begin{column}{6.5cm}
      
%       \begin{block}{Recall:}
%         \begin{itemize} 

%           \item Laplace's Equation in 2D:
%           \begin{equation}
%             \nonumber     
%             \frac{\partial^2 \phi}{\partial x^2} + \frac{\partial^2 \phi}{\partial y^2} = 0
%           \end{equation}

% 	  \item For the verification exercise, we will replace the RHS
% 	    above with a forcing function $f(x,y)$ that we get from
% 	    MASA
%           \begin{equation}
%             \nonumber     
%             \frac{\partial^2 \phi}{\partial x^2} + \frac{\partial^2
% 	      \phi}{\partial y^2} = f(x,y)
% 	    \end{equation}
          
%           % nick: not sure about this here, do we want to show mms?
%           % seems less useful to remind them about the source term
% %%           \item Our manufactured solution is:
% %%           \begin{equation}
% %%             \begin{split}
% %%               \nonumber
% %%               \phi(x,y) &= (\textcolor{blue}{Ly}-y)^2 (\textcolor{blue}{Ly}+y)^2 \\ 
% %%               &+ (\textcolor{red}{Lx}-x)^2 (\textcolor{red}{Lx}+x)^2
% %%             \end{split}
% %%           \end{equation}          
% %%           \item By default, \textcolor{blue}{Ly}$=0.82$ and \textcolor{red}{Lx}$=1.4$

%         \end{itemize}    
%       \end{block}
      
%     \end{column}
%   \end{columns}

% %%   % karl: can you check that this is the differencing scheme you intended?
% %%   Solve using forward difference:
% %%   \begin{equation}
% %%     \nonumber     
% %%     \phi_i'' \approx \frac{\phi_{i+2} - 2\phi_{i+1} + \phi_{i}}{h^2} + O(h^2)
% %%   \end{equation}
  
% \end{frame}

% %===============================================================================
% % add slide about laplace equation solution here
% %===============================================================================

% \begin{frame}
%   \frametitle{Manufactured solution to Laplace's Equation}
  
%   Laplace's Equation:
%   \begin{equation}
%       \nonumber      
%     \nabla^2 \phi = 0
%   \end{equation}
%   In two dimensions:
%   \begin{equation}
%       \nonumber      
%     \frac{\partial^2 \phi}{\partial x^2} + \frac{\partial^2 \phi}{\partial y^2} = 0
%   \end{equation}
%   \newline
%   \newline
%   ``Manufacture'' a solution, with two constants:
%   \begin{equation}
%       \nonumber
%     \phi(x,y) = (\textcolor{red}{Ly}-y)^2 (\textcolor{red}{Ly}+y)^2 + (\textcolor{red}{Lx}-x)^2 (\textcolor{red}{Lx}+x)^2
%   \end{equation}  

% \end{frame}


% \begin{frame}
%   \frametitle{Finite-difference Scheme}
%     \begin{block}{Method}
%       \begin{itemize} 
% 	\item Let us use a simple FD approximation for the Laplacian
%         \item Assume a constant spacing mesh for convenience
%         \item Central-differencing 
% 	  \begin{equation}
% 	    \nonumber     
% 	    \nabla^{2}{\phi}_{i,j} \approx \frac{\phi_{i+1,j} -
% 	      2\phi_{i,j} + \phi_{i-1,j}}{h^2} + 
% 	    \frac{\phi_{i,j+1} -
% 	      2\phi_{i,j} + \phi_{i,j-1}}{h^2} + O(h^2)
% 	  \end{equation}
% 	  \item Use this formula to build the coefficient entries into
% 	    a linear system $Ax=b$.  
% 	  \item The size of the linear system is the number of solution
% 	    points. Since we are on a square domain,  $N = npts*npts$
% 	  \item You may find it convenient to use a mapping
% 	    from a 2D index $\phi_{i,j}$ to a 1D index for the
% 	    solution vector of your linear system, $\phi_{\text{index}}$
% 	  \begin{equation}
% 	    \nonumber     
% 	    index = j+(i*npts);
% 	    \end{equation}
%       \end{itemize}
%     \end{block}
% \end{frame}


% \begin{frame}
%   \frametitle{Finite-difference Scheme}
%     \begin{block}{Boundary Conditions}
%       \begin{itemize} 
%           \item The 5-point FD stencil is incomplete on the boundaries of
% 	    our square domain 
% 	  \item We need to apply constraints to matrix
% 	    $A$ to enforce the Dirchlet conditions on the boundaries
% 	  \item Simplest method to enforce BCs:
% 	    \begin{itemize} 
% 	      \item zero out all matrix entries on the
% 		row associated with boundary point, $\phi_i$
% 	    \item set the
% 	      diagonal $A(i,i)$ = 1.0
% 	      \item Set the RHS function to the
% 		desired solution $f_i = \phi_{\text{exact}}$
% 		\end{itemize}
%       \end{itemize}
%     \end{block}
% \end{frame}

% \begin{frame}[fragile]
%   \frametitle{Finite-difference Scheme}
%     \begin{block}{Boundary Conditions}
%       \begin{itemize} 
%           \item Let's look at form of system matrix $A^*$ after BCs have
% 	    been applied for $npts=3$ (note $A=\frac{1}{h^2}A^*$):
% {\tiny
% \begin{verbatim}


%      j =     0      1      2      3      4      5      6      7      8 

% i =  0:   1.00   0.00   0.00   0.00   0.00   0.00   0.00   0.00   0.00 
% i =  1:   0.00   1.00   0.00   0.00   0.00   0.00   0.00   0.00   0.00 
% i =  2:   0.00   0.00   1.00   0.00   0.00   0.00   0.00   0.00   0.00 
% i =  3:   0.00   0.00   0.00   1.00   0.00   0.00   0.00   0.00   0.00 
% i =  4:   0.00   1.00   0.00   1.00  -4.00   1.00   0.00   1.00   0.00 
% i =  5:   0.00   0.00   0.00   0.00   0.00   1.00   0.00   0.00   0.00 
% i =  6:   0.00   0.00   0.00   0.00   0.00   0.00   1.00   0.00   0.00 
% i =  7:   0.00   0.00   0.00   0.00   0.00   0.00   0.00   1.00   0.00 
% i =  8:   0.00   0.00   0.00   0.00   0.00   0.00   0.00   0.00   1.00 


% \end{verbatim}
% }
% \item In this case, we only have {\em one} active interior solution point
%       \end{itemize}
%     \end{block}
% \end{frame}


% %===============================================================================
% % calculate the source term
% %===============================================================================

% \begin{frame}
%   \frametitle{Calculating the Source Term}

%   We insert our manufactured solution back into the governing equations:
%   \begin{equation}
%       \nonumber
%     \frac{\partial^2 ((Lx-x)^2 (Lx+x)^2)}{\partial x^2} + \frac{\partial^2 ((Ly-y)^2 (Ly+y)^2)}{\partial y^2} = 0
%   \end{equation}

%   \begin{equation}
%     \begin{split}
%       \nonumber      
%       = & 2(Lx-x)^2 - 8(Lx-x)(Lx+x)+2(Lx+x)^2 \\
%       + & 2(Ly-y)^2 - 8(Ly-y)(Ly+y)+2(Ly+y)^2 \\
%       \textcolor{red}{\neq} & \textcolor{red}{0}
%     \end{split}
%   \end{equation}
%   This does not satisfy Laplace's Equation!
%   \newline
%   \newline
%   \begin{block}{}
%     To balance the equation, add the residual to the RHS as a source term. 
%   \end{block}
% \end{frame}

% %===============================================================================
% % back to laplace: we have a solution, now what do we do with it?
% %===============================================================================
% \begin{frame}
%   \frametitle{Example Verification Use Case}
%   \begin{block}{}
%     To solve Laplace's Equation numerically, we need a discretization scheme.
%   \end{block}
%   Let's use a 2nd order finite central difference:
%   \begin{equation}
%       \nonumber      
%     \phi_i'' \approx \frac{\phi_{i+1} - 2\phi_{i} + \phi_{i-1}}{h^2} + O(h^2)
%   \end{equation}
%   This requires solving the implicit system of equations: 
%   \begin{equation}
%       \nonumber      
%       A \vec \phi = \textcolor{red}{f}
%   \end{equation}
%    You can use your favorite linear solver (e.g. PETSc) to solve the
%    system.
%  %, or we'll provide a simple (slow) Gauss-Seidel iterative example.
% %  We will use a conjugate gradient solver to invert the matrix A.

% \end{frame}

% %===============================================================================
% % application linkage
% %===============================================================================
% \begin{frame}
%   \frametitle{General Verification Approach Using MMS and MASA}
%   \begin{center}
%     \includegraphics[width=.8\linewidth]{masa_overview} \\
%   \end{center}
% \end{frame}

% %===============================================================================
% % back to laplace: we have a solution, now what do we do with it?
% %===============================================================================

%  \begin{frame}
%    \frametitle{Problem: Solve 2D Laplacian using Finite-Differencing}

%    \begin{block}{Outline}
%      \begin{itemize} 
%      \item {\em Goal:} Write a program in C/C++, F90
%      \item {\em Inputs}: 
%        \begin{itemize}
% 	 \item \# of points in one direction ({\em npts})
% 	 \item the physical dimension of one side ($L_x$, $L_y$)
%        \end{itemize}
%      \item {\em Output}: $l_2$ error between your numerical solution
%        and an exact solution derived from a manufactured solution
%        \begin{equation}
%          \nonumber
%          l_2 = \sqrt{ \frac{\sum_{i=1}^{\text{\tiny N}} (\phi_i-\phi_i^{\text{\tiny exact}})^2}N}
%        \end{equation}
%      \item {\em Runs}: Run your snazzy code for $\text{npts} =
%        5,9,17,\text{and } 33$ and plot $l_2$ norm as a function of $1/h$ where
%        $h=\text{length}/(\text{npts}-1)$
       
%      \end{itemize}    
%    \end{block}

%  \end{frame}

% %% %===============================================================================
% %% % fix bug
% %% %===============================================================================
%  \begin{frame}
%    \frametitle{This Process Finds Bugs}
%       \begin{center}
%         \includegraphics[scale=.8]{mms_grid_convergence.pdf} \\
%       \end{center}

%  \end{frame}
	   
% % Nick to do
% \begin{frame}
%   \frametitle{Installing MASA locally}
%   \begin{block}{Steps for Building MASA:}
%     \begin{itemize}
%       \item Grab latest tarball (https://red.ices.utexas.edu/attachments/download/1352/masa-0.40.2.tar.gz)
%         % bit of a mouthful but thats how it is
%         % https://red.ices.utexas.edu/projects/software/files
%       \item Untar: tar xvfz masa-0.40.2.tar.gz 
%       \item Configure: ./configure --prefix=\$HOME/masa % what about compilers here?
%       \item Compile: make -j 2
%       \item Test: make check
%       \item Install locally: make install
%       \item To generate documenations: make docs 
% 	\begin{itemize}
% 	  \item Can then point a broswer to docs/html/index.html
% 	\end{itemize}
%     \end{itemize}
%    \end{block}

% \end{frame}

% \begin{frame}[fragile]
%   \frametitle{Linking to your installed MASA}
%   \begin{block}{Linking against your local build}
%     \begin{itemize} 

%       \item {\bf C}: Assuming your code is named laplacian.c and you
%       installed masa into \$HOME/masa: 
% {\tiny
% \begin{verbatim}

% gcc -I$HOME/masa/include laplacian.c -L$HOME/masa/lib -lmasa

% \end{verbatim}
% }
      
%       \item {\bf F90}: Assuming your code is named laplacian.f90 
% {\tiny
% \begin{verbatim}

% gfortran -I$HOME/masa/lib laplacian.f90 -L$HOME/masa/lib -lmasa -lfmasa

% \end{verbatim}
% }
%       \end{itemize}
%     \end{block}
% \end{frame}

% %===============================================================================
% % refresher
% %===============================================================================
% \begin{frame}
%   \frametitle{Problem: Solve 2D Laplacian using Finite-Differencing}
%   \begin{columns}[c]
%     \begin{column}{5cm}

%       \includegraphics[width=1\linewidth]{domain}

%     \end{column}
    
%     \begin{column}{6.5cm}
      
%       \begin{block}{Recall:}
%         \begin{itemize} 

%           \item Laplace's Equation in 2D:
%           \begin{equation}
%             \nonumber     
%             \frac{\partial^2 \phi}{\partial x^2} + \frac{\partial^2 \phi}{\partial y^2} = 0
%           \end{equation}

% 	  \item For the verification exercise, we will replace the RHS
% 	    above with a forcing function $f(x,y)$ that we get from
% 	    MASA
%           \begin{equation}
%             \nonumber     
%             \frac{\partial^2 \phi}{\partial x^2} + \frac{\partial^2
% 	      \phi}{\partial y^2} = f(x,y)
% 	    \end{equation}
          
%         \end{itemize}    
%       \end{block}
      
%     \end{column}
%   \end{columns}

% \end{frame}

% \begin{frame}
%   \frametitle{Finite-difference Scheme}
%     \begin{block}{Method}
%       \begin{itemize} 
% 	\item Let us use a simple FD approximation for the Laplacian
%         \item Assume a constant spacing mesh for convenience
%         \item Central-differencing 
% 	  \begin{equation}
% 	    \nonumber     
% 	    \nabla^{2}{\phi}_{i,j} \approx \frac{\phi_{i+1,j} -
% 	      2\phi_{i,j} + \phi_{i-1,j}}{h^2} + 
% 	    \frac{\phi_{i,j+1} -
% 	      2\phi_{i,j} + \phi_{i,j-1}}{h^2} + O(h^2)
% 	  \end{equation}
% 	  \item Use this formula to build the coefficient entries into
% 	    a linear system $Ax=b$.  
% 	  \item The size of the linear system is the number of solution
% 	    points. Since we are on a square domain,  $N = npts*npts$
% 	  \item You may find it convenient to use a mapping
% 	    from a 2D index $\phi_{i,j}$ to a 1D index for the
% 	    solution vector of your linear system, $\phi_{\text{index}}$
% 	  \begin{equation}
% 	    \nonumber     
% 	    index = j+(i*npts);
% 	    \end{equation}
%       \end{itemize}
%     \end{block}
% \end{frame}



% %===============================================================================
% % application linkage
% %===============================================================================
% \begin{frame}[fragile]
% \frametitle{Fortran 90: What you need from MASA}
% {\tiny
% \begin{verbatim}
% program main
%   use masa
%   implicit none

%   dx = real(lx)/real(nx)
%   dy = real(ly)/real(ny);

%   ! initialize the problem
%   call masa_init("laplace example","laplace_2d")

%   ! evaluate source terms (2D)
%   do i=0, nx
%      do j=0, ny
         
%         y = j*dy        
%         x = i*dx
        
%         ! evalulate source term
%         field = masa_eval_2d_source_f   (x,y)

%         ! evaluate analytical term
%         exact_phi = masa_eval_2d_exact_phi (x,y)

%      enddo
%   enddo

% end program main

% \end{verbatim}
% }
% \end{frame}

% %===============================================================================
% % MASA API demo
% %===============================================================================
% \begin{frame}[fragile]
% \frametitle{C: What you need from MASA}
% {\tiny
% \begin{verbatim}
% #include <masa.h>

% int main()
% {
%   err += masa_init("laplace example","laplace_2d");

%   // grab / set parameter values
%   Lx = masa_get_param("Lx");
%   masa_set_param("Ly",42.0);

%   for(int i=0;i<nx;i++)
%     for(int j=0;j<nx;j++)
%       {  
%         x=i*dx;
%         y=j*dy;

%         // source term
%         ffield    = masa_eval_2d_source_f (x,y);

%         // manufactured solution
%         phi_field = masa_eval_2d_exact_phi(x,y);

%        } // finished iterating over space
% } //end program
% \end{verbatim}
% }
% \end{frame}

% %===============================================================================
% % convergence plot
% %===============================================================================
% \begin{frame}
%   \frametitle{Example Results: What we're hoping for}
%   2nd Order Central Finite-difference Scheme

%  \begin{center}
%   \includegraphics[width=0.7\linewidth]{laplace_central_diff2}
%  \end{center}
  
% \end{frame}

% %===============================================================================
% % conclusions
% %===============================================================================

%  \begin{frame}
%   \frametitle{Conclusions}

%    \begin{block}{}
%     \begin{center}
%      Good Luck!
%     \end{center}
%     \center{Have a well verified day.} \\
%     \center{nick@ices.utexas.edu}
%     \end{block}
%  \end{frame}

\end{document}
