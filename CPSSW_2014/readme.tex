Notes for the talk from the Computational Physics Student Summer Workshop. 


Notes: 

I know you are busy and squeezing in your lecture preparation time is a
challenge.  At the risk of adding more burden to your task, I'd like to
ask that you try to make pedagogy a priority as you prepare your
lectures.  I am still hoping we can transform what we've called our
"Lectures Series" of the past into a college-credit worthy "Course" in
the future.    
 
Below are some specific requests I am making:
 
(1) When appropriate, refer to my ContinuumTransport lecture , using
slides or terminology from it as needed.  There is no need to re-derive
conservation or generic transport concepts.  Rather, building on what
they've seen so far (and they have been asked to study and master), will
be more efficient and hopefully easier on you. 
 
(2) Try to adopt some sort of pedagogical model.  It is best, in my
opinion, to start very simply and gradually build up.  No one will grow
impatient if you start below their knowledge level. 
 
(3) Please define all symbols in equations that you use.  Go over the
symbols slowly and carefully. 